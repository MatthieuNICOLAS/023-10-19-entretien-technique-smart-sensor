\section{Intégration des opérations \ins et \rmv concurrentes - détails}

\begin{frame}[fragile]{Opération \ren}
    \begin{figure}
      \resizebox{\textwidth}{!}{
        \begin{tikzpicture}
          \newcommand\nodeh[1]{
              node[letter, label=#1:{$\id{i}{B1}{0}$}] {H}
          }
          \newcommand\nodee[1]{
              node[letter, fill=\colorblockone, label=#1:{$\coloridone\id{i}{B1}{0}\id{f}{A1}{0}$}] {E}
          }
          \newcommand\nodelo[1]{
              node[block, label=#1:{$\id{i}{B1}{1..2}$}] {LO}
          }
          \newcommand\renh[1]{
              node[letter, fill=\colorblocktwo, label=#1:{$\coloridtwo\id{i}{A2}{0}$}] {H}
          }
          \newcommand\rene[1]{
              node[letter, fill=\colorblocktwo, label=#1:{$\coloridtwo\id{i}{A2}{1}$}] {E}
          }
          \newcommand\renl[1]{
              node[letter, fill=\colorblocktwo, label=#1:{$\coloridtwo\id{i}{A2}{2}$}] {L}
          }
          \newcommand\reno[1]{
              node[letter, fill=\colorblocktwo, label=#1:{$\coloridtwo\id{i}{A2}{3}$}] {O}
          }
          \newcommand\renhelo[1]{
              node[block, fill=\colorblocktwo, label=#1:{$\coloridtwo\id{i}{A2}{0..3}$}] {HELO}
          }

          \newcommand\initialstate[3]{
              \path
              #1
              ++#2
              ++(0:0.5)
              ++(#3:0.5) \nodeh{-#3}
              ++(0:\widthletter) \nodee{#3}
              ++(0:\widthletter) \nodelo{-#3};
          }

          \newcommand\stateh[3]{
              \path
              #1
              ++#2
              ++(0:0.5)
              ++(#3:0.5) \renh{-#3};
          }

          \newcommand\statehe[3]{
            \path
            #1
            ++#2
            ++(0:0.5)
            ++(#3:0.5) \renh{-#3}
            ++(0:\widthletter) \rene{-#3};
          }

          \newcommand\statehel[3]{
            \path
            #1
            ++#2
            ++(0:0.5)
            ++(#3:0.5) \renh{-#3}
            ++(0:\widthletter) \rene{-#3}
            ++(0:\widthletter) \renl{-#3};
          }

          \newcommand\statehelo[3]{
            \path
            #1
            ++#2
            ++(0:0.5)
            ++(#3:0.5) \renh{-#3}
            ++(0:\widthletter) \rene{-#3}
            ++(0:\widthletter) \renl{-#3}
            ++(0:\widthletter) \reno{-#3};
          }

          \newcommand\finalstate[3]{
            \path
            #1
            ++#2
            ++(0:0.5)
            ++(#3:0.5) \renhelo{-#3};
          }

          \newcommand\offseta{ (90:0.7) }

          \path
              node {\textbf{A}}
              ++(0:0.5) node (a) {}
              +(0:11) node (a-end) {}
              +(0:1) node[point] (a-initial) {}
              +(0:6) node[point] (a-ren-a1) {}
              +(0:10) node (a-final) {};

          \initialstate{(a-initial)}{\offseta}{90};

          \path (a-ren-a1) node[label=-170:{$\trm{ren}()$}, label={[xshift=0pt]-10:{$\trm{ren}(A,2)$}}] {};

          \finalstate{(a-ren-a1)}{\offseta}{90};
          \path (a-ren-a1) node[label=-170:{$\trm{ren}()$}, label={[xshift=0pt]-10:{$\trm{ren}(A,2,\trm{renIds})$}}] {};

          \draw[dotted] (a) -- (a-initial) (a-final) -- (a-end);
          \draw[->, thick] (a-initial) --  (a-ren-a1) -- (a-final);
        \end{tikzpicture}
      }
    \end{figure}
    \begin{itemize}
      \item Génère nouvel identifiant pour le 1er élément : $\id{i}{B1}{0} \to \id{i}{A2}{0}$
      \item Puis génère identifiants contigus pour éléments suivants : $\id{i}{A2}{1}$, $\id{i}{A2}{2}$, \dots
    \end{itemize}

    \begin{center}
      \alert{Regroupe tous les éléments en 1 unique bloc}
    \end{center}

    \begin{block}{Pour plus tard :}
      \begin{itemize}
        \item Stocke identifiants ($[\id{i}{B1}{0},\id{i}{B1}{0}\id{f}{A1}{0},\dots]$) de l'état d'origine : $\trm{renIds}$
      \end{itemize}
  \end{block}
\end{frame}

\begin{frame}{Mécanisme de résolution de conflits entre une opération \ren et une opération \ins ou \rmv}
    \begin{block}{Besoins}
        \begin{enumerate}
            \item Détecter les opérations concurrentes aux opérations \ren
            \item Prendre en compte effet des opérations \ren lors de l'intégration des opérations concurrentes
        \end{enumerate}
    \end{block}
\end{frame}

\begin{frame}[fragile]{Détection des opérations concurrentes à opération \ren}
    % \begin{center}
    %     \alert{Besoin de détecter opérations concurrentes à opération \ren}
    % \end{center}
    \vspace{-1.5em}
    \begin{figure}
        \resizebox{\textwidth}{!}{
          \begin{tikzpicture}
            \newcommand\nodeh[1]{
                node[letter, label=#1:{$\id{i}{B1}{0}$}] {H}
            }
            \newcommand\nodee[1]{
                node[letter, fill=\colorblockone, label=#1:{\coloridone$\id{i}{B1}{0}\id{f}{A1}{0}$}] {E}
            }
            \newcommand\nodelo[1]{
                node[block, label=#1:{$\id{i}{B1}{1..2}$}] {LO}
            }
            \newcommand\renhelo[1]{
                node[block, fill=\colorblocktwo, label=#1:{\coloridtwo$\id{i}{A2}{0..3}$}] {HELO}
            }
            \newcommand\nodel[1]{
                node[letter, fill=\colorblockthree,label=#1:{\coloridthree$\id{i}{B1}{0}\id{m}{B2}{0}$}] {L}
            }
            \newcommand\renhe[1]{
                node[block, fill=\colorblocktwo, label=#1:{\coloridtwo$\id{i}{A2}{0..1}$}] {HE}
            }
            \newcommand\renl[1]{
                node[letter, fill=\colorblockfour, label=#1:{\coloridfour$\id{i}{A2}{1}\id{i}{B1}{0}\id{m}{B2}{0}$}] {L}
            }
            \newcommand\renlo[1]{
                node[block, fill=\colorblocktwo, label=#1:{\coloridtwo$\id{i}{A2}{2..3}$}] {LO}
            }

            \newcommand\initialstate[3]{
                \path
                    #1
                    ++#2
                    ++(0:0.5)
                    ++(#3:0.5)
                    ++(0:1.25 * \widthoriginepoch) \nodeh{-#3}
                    ++(0:\widthletter) \nodee{#3}
                    ++(0:\widthletter) \nodelo{-#3};
            }

            \newcommand\initialstatewithepoch[3]{
                \path
                    #1
                    ++#2
                    ++(0:0.5)
                    ++(#3:0.5) node[epoch] {$\epoch{0}$}
                    ++(0:1.25 * \widthoriginepoch) \nodeh{-#3}
                    ++(0:\widthletter) \nodee{#3}
                    ++(0:\widthletter) \nodelo{-#3};
            }

            \newcommand\statehelo[3]{
                \path
                    #1
                    ++#2
                    ++(0:0.5)
                    ++(#3:0.5)
                    ++(0:1.18 * \widthepoch) \renhelo{-#3};
            }

            \newcommand\statehelowithepoch[3]{
                \path
                    #1
                    ++#2
                    ++(0:0.5)
                    ++(#3:0.5) node[epoch] {$\epoch{A2}$}
                    ++(0:1.18 * \widthepoch) \renhelo{-#3};
            }

            \newcommand\insl[3]{
                \path
                #1
                ++#2
                ++(0:0.5)
                ++(#3:0.5)
                ++(0:1.25 * \widthoriginepoch) \nodeh{-#3}
                ++(0:\widthletter) \nodee{#3}
                ++(0:\widthletter) \nodel{-#3}
                ++(0:\widthletter) \nodelo{#3};
            }

            \newcommand\inslwithepoch[3]{
                \path
                #1
                ++#2
                ++(0:0.5)
                ++(#3:0.5) node[epoch] {$\epoch{0}$}
                ++(0:1.25 * \widthoriginepoch) \nodeh{-#3}
                ++(0:\widthletter) \nodee{#3}
                ++(0:\widthletter) \nodel{-#3}
                ++(0:\widthletter) \nodelo{#3};
            }

            \newcommand\offseta{ (90:0.7) }
            \newcommand\offsetb{ (-90:0.7) }

            \path
                node {\textbf{A}}
                ++(0:0.5) node (a) {}
                +(0:15) node (a-end) {}
                +(0:1) node[point] (a-initial) {}
                +(0:7) node[point, label=-170:{$\trm{ren}()$}, ] (a-ren-a1) {}
                +(0:13) node[point] (a-recv-ins-l) {}
                +(0:14) node (a-final) {};
            \path
                (a-ren-a1) node[label={[xshift=0pt]-10:{$\trm{ren}(\epoch{0}, A,2,\renids{A2})$}}] {};

            \initialstatewithepoch{(a-initial)}{\offseta}{90};
            \statehelowithepoch{(a-ren-a1)}{\offseta}{90};

            \draw[dotted] (a) -- (a-initial) (a-final) -- (a-end);
            \draw[->, thick] (a-initial) --  (a-ren-a1) -- (a-recv-ins-l) -- (a-final);

            \path
                ++(270:2) node {\textbf{B}}
                ++(0:0.5) node (b) {}
                +(0:15) node (b-end) {}
                +(0:1) node[point] (b-initial) {}
                +(0:7) node[point, label=170:{$\trm{ins}(E \prec L \prec L)$}] (b-ins-l) {}
                +(0:14) node (b-final) {};
            \path
                (b-ins-l) node[label={[xshift=45pt]10:{$\trm{ins}(\epoch{0}, {\coloridthree\id{i}{B1}{0}\id{m}{B2}{0}},L)$}}] {};

            \initialstatewithepoch{(b-initial)}{\offsetb}{-90};
            \inslwithepoch{(b-ins-l)}{\offsetb}{-90};

            \draw[dotted] (b) -- (b-initial) (b-final) -- (b-end);
            \draw[->, thick] (b-initial) --  (b-ins-l) -- (b-final);

            \draw[->, dashed, shorten >= 1] (b-ins-l) -- (a-recv-ins-l);
          \end{tikzpicture}
        }
    \end{figure}
    \metroset{block=transparent}
    \begin{block}{Ajout mécanisme d'époques}
        \begin{itemize}
            \item Séquence commence à époque d'origine, notée $\epoch{0}$
            \item \ren font progresser à nouvelle époque, $\epoch{\trm{nodeId}~\trm{nodeSeq}}$
            \item Opérations labellisées avec époque de génération
        \end{itemize}
    \end{block}
\end{frame}

\begin{frame}[fragile]{Intégration des opérations \ins et \rmv concurrentes}
  \vspace{-0.5 em}
  \begin{figure}
    \resizebox{0.9 \textwidth}{!}{
      \begin{tikzpicture}
        \newcommand\nodeh[1]{
            node[letter, label=#1:{$\id{i}{B1}{0}$}] {H}
        }
        \newcommand\nodee[1]{
            node[letter, fill=\colorblockone, label=#1:{\coloridone$\id{i}{B1}{0}\id{f}{A1}{0}$}] {E}
        }
        \newcommand\nodelo[1]{
            node[block, label=#1:{$\id{i}{B1}{1..2}$}] {LO}
        }
        \newcommand\renhelo[1]{
            node[block, fill=\colorblocktwo, label=#1:{\coloridtwo$\id{i}{A2}{0..3}$}] {HELO}
        }
        \newcommand\nodel[1]{
            node[letter, fill=\colorblockthree,label=#1:{\coloridthree$\id{i}{B1}{0}\id{m}{B2}{0}$}] {L}
        }
        \newcommand\renhe[1]{
            node[block, fill=\colorblocktwo, label=#1:{\coloridtwo$\id{i}{A2}{0..1}$}] {HE}
        }
        \newcommand\renl[1]{
            node[letter, fill=\colorblockfour, label=#1:{\coloridfour$\id{i}{A2}{1}\id{i}{B1}{0}\id{m}{B2}{0}$}] {L}
        }
        \newcommand\renlo[1]{
            node[block, fill=\colorblocktwo, label=#1:{\coloridtwo$\id{i}{A2}{2..3}$}] {LO}
        }

        \newcommand\statehelo[3]{
            \path
                #1
                ++#2
                ++(0:0.5)
                ++(#3:0.5) node[epoch] {$\epoch{A2}$}
                ++(0:1.18 * \widthepoch) \renhelo{-#3};
        }

        \newcommand\insl[3]{
            \path
            #1
            ++#2
            ++(0:0.5)
            ++(#3:0.5) node[epoch] {$\epoch{0}$}
            ++(0:1.25 * \widthoriginepoch) \nodeh{-#3}
            ++(0:\widthletter) \nodee{#3}
            ++(0:\widthletter) \nodel{-#3}
            ++(0:\widthletter) \nodelo{#3};
        }

        \newcommand\statehellowithoutid[3]{
            \path
            #1
            ++#2
            ++(0:0.5)
            ++(#3:0.5) node[epoch] {$\epoch{A2}$}
            ++(0:1.18 * \widthepoch) \renhe{-#3}
            ++(0:\widthblock) node[letter, fill=\colorblockfour, label=#3:{\coloridfour ?}] {L}
            ++(0:\widthletter) \renlo{-#3};
        }

        \newcommand\finalstate[3]{
            \path
                #1
                ++#2
                ++(0:0.5)
                ++(#3:0.5) node[epoch] {$\epoch{A2}$}
                ++(0:1.18 * \widthepoch) \renhe{-#3}
                ++(0:\widthblock) \renl{#3}
                ++(0:\widthletter) \renlo{-#3};
        }

        \newcommand\offseta{ (90:0.7) }
        \newcommand\offsetb{ (-90:0.7) }

        \path
            node {\textbf{A}}
            ++(0:0.5) node (a) {}
            +(0:13) node (a-end) {}
            +(0:1) node[point] (a-initial) {}
            +(0:7) node[point] (a-recv-ins-l) {}
            +(0:12) node (a-final) {};

        \statehelo{(a-initial)}{\offseta}{90};
        \finalstate{(a-recv-ins-l)}{\offseta}{90};

        \draw[dotted] (a) -- (a-initial) (a-final) -- (a-end);
        \draw[->, thick] (a-initial) -- (a-recv-ins-l) -- (a-final);

        \path
            ++(270:2) node {\textbf{B}}
            ++(0:0.5) node (b) {}
            +(0:13) node (b-end) {}
            +(0:1) node[point, label={[xshift=45pt]10:{$\trm{ins}(\epoch{0}, {\coloridthree\id{i}{B1}{0}\id{m}{B2}{0}},L)$}}] (b-initial) {}
            +(0:12) node (b-final) {};

        \insl{(b-initial)}{\offsetb}{-90};

        \draw[dotted] (b) -- (b-initial) (b-final) -- (b-end);
        \draw[->, thick] (b-initial) -- (b-final);

        \draw[->, dashed, shorten >= 1] (b-initial) -- (a-recv-ins-l);
      \end{tikzpicture}
    }
  \end{figure}
  \vspace{-1em}
  \textbf{Rappel : }
  \begin{equation*}
      \renids{A2} = [\id{i}{B1}{0},\id{i}{B1}{0}\id{f}{A1}{0},\id{i}{B1}{1},\id{i}{B1}{2}]
  \end{equation*}
  \vspace{-1em}
  \begin{block}{Exemple avec {$\coloridone\id{i}{B1}{0}\id{m}{B2}{0}$}}
    \begin{itemize}
      \item Rechercher son prédecesseur dans $\renids{A2}$ : $\coloridone\id{i}{B1}{0}\id{f}{A1}{0}$
      \item Utiliser son index ($\coloridtwo 1$) pour calculer équivalent à époque $\epoch{A2}$ : $\coloridtwo\id{i}{A2}{1}$
      \item Préfixer {$\coloridthree\id{i}{B1}{0}\id{m}{B2}{0}$} par ce dernier : $\coloridfour\id{i}{A2}{1}$$\coloridthree\id{i}{B1}{0}\id{m}{B2}{0}$
    \end{itemize}
  \end{block}
\end{frame}
