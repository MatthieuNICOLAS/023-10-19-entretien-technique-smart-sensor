\subsection{Validation}

\begin{frame}{Validation}
  \begin{itemize}
    \item \alert{Montrer que RenamableLogootSplit satisfait la cohérence forte à terme}
    \item \alert{Montrer que le mécanisme de renommage améliore les performances} de la séquence répliquée (mémoire, calculs, bande-passante)
  \end{itemize}
  \pause
  \begin{center}
    \alert{Conduite d'une évaluation expérimentale}
  \end{center}
\end{frame}

\begin{frame}[standout]
  \alert{Absence d'un jeu de données de sessions d'édition collaborative}

  \medskip
  \pause
  Mise en place de simulations pour générer un jeu de données
\end{frame}

\begin{frame}{Simulations - Principe}
  \metroset{block=transparent}

  \begin{itemize}
    \item Simulation d'une session d'édition collaborative
    \item \alert{10 noeuds} communiquant via un \alert{réseau entièrement maillé}
    \item Noeuds \alert{effectuent modifications périodiquement} et \alert{intègrent modifications reçues} dès que possible, \alert{sans coordination}
    \item \alert{2 phases} : génération de contenu (80\% d'\ins, 20\% de \rmv) puis édition (50/50\%)
  \end{itemize}
\end{frame}

\begin{frame}{Simulations - Sorties \& résultats}
  \metroset{block=transparent}
  \begin{itemize}
    \item \alert{Instantané de l'état} de chaque noeud à différents points de la simulation (10k opérations et état final)
    \item \alert{Journal des opérations} de chaque noeud
  \end{itemize}
  \pause

  \begin{block}{Conduite d'évaluations sur ces données\singlefootnote{Code des simulations et benchmarks : \url{https://github.com/coast-team/mute-bot-random}}}
    \begin{itemize}
      \item Validation de l'amélioration des performances de la séquence répliquée (mémoire, calculs, bande-passante)
    \end{itemize}
  \end{block}

\end{frame}
