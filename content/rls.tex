\section{RenamableLogootSplit}

\subsection{Réplication de données dans systèmes pair-à-pair}

\begin{frame}{MUTE}
    \vspace{-0.5cm}
    \begin{figure}
        \resizebox{\textwidth}{!}{
            \includegraphics{img/screenshot-mute-editor.png}
        }
    \end{figure}
    \vspace{-0.5cm}
    \begin{itemize}
        \item Application pair-à-pair
        \item Permet de rédiger collaborativement des documents texte
        \item Garantit la confidentialité \& souveraineté des données
    \end{itemize}
\end{frame}

\begin{frame}[fragile]{Réplication dans applications collaboratives pair-à-pair}
    \begin{figure}
        \resizebox{0.7 \textwidth}{!}{
            \begin{tikzpicture}
                \newcommand{\doc}{
                    \tikz{
                        \fill[scale=.15,fill=white,draw=gray,thick,solid] (0,0) -- (7,0) -- (7,8) -- (5,10) -- (0,10) -- cycle;
                    }
                }
                \newcommand{\updsquare}{
                    \tikz{
                        \fill[\colorblockone, scale=.12] (0,0) rectangle (3,3);
                    }
                }
                \newcommand{\updcircle}{
                    \tikz{
                        \fill[\colorblocktwo, scale=.07] (3,3) circle (3);
                    }
                }
                \newcommand{\updtriangle}{
                    \tikz{
                        \fill[\colorblockfive, scale=.07] (0,0) -- (6,0) -- (3,6) -- cycle;
                    }
                }
                \path
                    node[label=90:{A}] (a) {
                        \includegraphics[scale=0.4, page=5, trim=0cm 24cm 32cm 0cm, clip]{img/mute-figures.pdf}
                    }
                    +(-70:3) node[label=-90:{B}] (b) {
                        \includegraphics[scale=0.4, page=5, trim=0cm 24cm 32cm 0cm, clip]{img/mute-figures.pdf}
                    }
                    +(200:4) node[label=-90:{C}] (c) {
                        \includegraphics[scale=0.4, page=5, trim=0cm 24cm 32cm 0cm, clip]{img/mute-figures.pdf}
                    };

                \path
                    (a) node[label={[xshift=2em]0:{\doc}}] {}
                    (b) node[label={[xshift=2em]0:{\doc}}] {}
                    (c) node[label={[xshift=-2em]180:{\doc}}] {};

                \onslide<4->{
                    \path
                        (a) node[label={[xshift=3.8em]90:{\updsquare}}] {}
                        (b) node[label={[xshift=3.7em]0:{\updtriangle}}] {}
                        (c) node[label={[xshift=-4.8em]-90:{\updcircle}}] {};
                }

                \onslide<5->{
                    \draw[dotted] (a) -- (b);
                }

                \only<6>{
                    \path
                        (a) -- node[midway]{\includegraphics[scale=0.4]{img/sync.pdf}} (b);
                }

                \onslide<7->{
                    \path
                        (a) node[label={[xshift=3.7em]0:{\updtriangle}}] {}
                        (b) node[label={[xshift=3.8em]90:{\updsquare}}] {};
                }

                \onslide<8->{
                    \draw[dotted] (a) -- (c);
                    \draw[dotted] (b) -- (c);
                }

                \only<8> {
                    \path
                        (a) -- node[midway]{\includegraphics[scale=0.4]{img/sync.pdf}} (c)
                        (b) -- node[midway]{\includegraphics[scale=0.4]{img/sync.pdf}} (c);
                }

                \onslide<9->{
                    \path
                        (a) node[label={[xshift=3.8em]-90:{\updcircle}}] {}
                        (b) node[label={[xshift=3.8em]-90:{\updcircle}}] {}
                        (c) node[label={[xshift=-4.8em]90:{\updsquare}}] {}
                        (c) node[label={[xshift=-4.8em]0:{\updtriangle}}] {};
                }
            \end{tikzpicture}
        }
    \end{figure}
    \vspace{-1em}
    \begin{columns}
        \hspace{0em}
        \begin{column}{0.6\textwidth}
            \begin{itemize}
                \item<2-> Noeuds peuvent être \alert{déconnectés}
                \item<3-> Doivent pouvoir \alert{travailler sans coordination synchrone} préalable (par ex. consensus)
            \end{itemize}
        \end{column}
        \begin{column}{0.6\textwidth}
            \begin{itemize}
                \item<9-> Doit garantir \alert{cohérence à terme} \cite{10.1145/224057.224070}\dots
                \item<9-> \dots malgré ordres différents d'intégration des modifications
            \end{itemize}
        \end{column}
    \end{columns}
    \onslide<10>{
        \vspace{1em}
        \begin{center}
            \alert{Nécessite des \emph{mécanismes de résolution de conflits}}
        \end{center}
    }
\end{frame}

% \begin{frame}{Évaluation de MUTE}
%     \metroset{block=transparent}
%     \begin{block}{Taille du texte comparée à taille de la séquence répliquée}
%         \begin{columns}
%             \begin{column}{0.6\textwidth}
%                 \begin{figure}
%                     \resizebox{\columnwidth}{!}{
%                         \includegraphics{img/ls-vs-content-snapshot-sizes-7k5.pdf}
%                     }
%                 \end{figure}
%             \end{column}
%             \begin{column}{0.4\textwidth}
%                 \pause
%                 \begin{block}{Constat}
%                     \begin{itemize}
%                         \item 1\% contenu\dots
%                         \item \dots 99\% métadonnées
%                     \end{itemize}
%                 \end{block}
%                 \pause
%                 \begin{center}
%                     \alert{Et ça augmente !}
%                 \end{center}
%                 \pause
%                 \begin{block}{Impact}
%                     \begin{itemize}
%                         \item Surcoût \alert{mémoire}\dots
%                         \item \dots mais aussi surcoût en \alert{calculs} et en \alert{bande-passante}
%                     \end{itemize}
%                 \end{block}
%             \end{column}
%         \end{columns}
%     \end{block}
% \end{frame}

% \begin{frame}[standout]
%     Comment peut-on \alert{réduire le surcoût} des mécanismes de résolution de conflits dans les applications pair-à-pair ?
% \end{frame}

\subsection{Conflict-free Replicated Data Types (CRDTs) pour le type Séquence}

\begin{frame}{Conflict-free Replicated Data Types (CRDTs)  \cite{shapiro_2011_crdt}}
    \begin{itemize}
        \item Nouvelles spécifications des types de données, \eg \emph{Ensemble} ou \emph{Séquence}
        \item Incorpore nativement mécanisme de résolution de conflits
    \end{itemize}
    \pause
    \begin{block}{Propriétés des CRDTs}
        \begin{itemize}
            \item Permettent modifications \alert{sans coordination}
            \item Garantissent la \alert{cohérence forte à terme}
        \end{itemize}
    \end{block}
    \pause
    \begin{block}{Cohérence forte à terme}
        Ensemble des noeuds ayant intégrés le même ensemble de modifications obtient des états équivalents, \alert{sans nécessiter d'actions ou messages supplémentaires}
    \end{block}
\end{frame}

\begin{frame}[fragile]{CRDTs pour le type Séquence}
    \metroset{block=transparent}
    \begin{columns}
        \begin{column}{0.6\textwidth}
            \begin{block}{\quad Type Séquence usuel}
                \begin{figure}[!ht]
                    \centering
                    \resizebox{\columnwidth}{!}{
                        \begin{tikzpicture}
                            \newcommand\initialstate[2]{
                                \path
                                    #1
                                    ++#2
                                    node[letter, label=below:{$0$}] {B}
                                    ++(0:\widthletter) node[letter, label=below:{$1$}] {N}
                                    ++(0:\widthletter) node[letter, label=below:{$2$}] {J}
                                    ++(0:\widthletter) node[letter, label=below:{$3$}] {O};
                            }
                            \newcommand\totoa[2]{
                                \path
                                    #1
                                    ++#2
                                    node[letter, label=below:{$0$}] {}
                                    ++(0:\widthletter) node[letter, label=below:{$1$}] {}
                                    ++(0:\widthletter) node[letter, label=below:{$2$}] {}
                                    ++(0:\widthletter) node[letter, label=below:{$3$}] {}
                                    ++(0:\widthletter) node[letter, label=below:{$4$}] {};
                            }
                            \newcommand\totob[2]{
                                \path
                                    #1
                                    ++#2
                                    node[letter, label=below:{$0$}] {B}
                                    ++(0:\widthletter) node[letter, label=below:{$1$}] {}
                                    ++(0:\widthletter) node[letter, label=below:{$2$}] {N}
                                    ++(0:\widthletter) node[letter, label=below:{$3$}] {J}
                                    ++(0:\widthletter) node[letter, label=below:{$4$}] {O};
                            }
                            \newcommand\totoc[2]{
                                \path
                                    #1
                                    ++#2
                                    node[letter, label=below:{$0$}] {B}
                                    ++(0:\widthletter) node[letter, label=below:{$1$}] {A}
                                    ++(0:\widthletter) node[letter, label=below:{$2$}] {N}
                                    ++(0:\widthletter) node[letter, label=below:{$3$}] {J}
                                    ++(0:\widthletter) node[letter, label=below:{$4$}] {O};
                            }

                            \newcommand\offseta{ (90:1) }

                            \path
                                node {\textbf{A}}
                                ++(0:0.5) node (a) {}
                                +(0:10) node (a-end) {}
                                +(0:1) node[point] (a-initial) {}
                                +(0:6) node (a-ins-a) {}
                                +(0:9) node (a-final) {};

                            \initialstate{(a-initial)}{\offseta};
                            \draw[dotted] (a) -- (a-initial) (a-final) -- (a-end);
                            \only<1>{
                                \draw[->, thick] (a-initial) -- (a-final);
                            }
                            \onslide<2->{
                                \path (a-ins-a) node[point, label=-170:{$\trm{ins}(1, A)$}] {};
                                \draw[->, thick] (a-initial) -- (a-ins-a) -- (a-final);
                            }
                            \only<3>{\totoa{(a-ins-a)}{\offseta}}
                            \only<4>{\totob{(a-ins-a)}{\offseta}}
                            \onslide<5->{\totoc{(a-ins-a)}{\offseta}}
                        \end{tikzpicture}
                    }
                \end{figure}
            \end{block}
        \end{column}
        \begin{column}{0.6\textwidth}
            \onslide<7->{
                \begin{block}{CRDTs pour Séquence}
                    \begin{figure}[!ht]
                        \centering
                        \resizebox{\columnwidth}{!}{
                            \begin{tikzpicture}
                                \newcommand\initialstate[2]{
                                    \path
                                        #1
                                        ++#2
                                        node[letter, label=below:{$\trm{id}_0$}] {B}
                                        ++(0:\widthletter) node[letter, label=below:{$\trm{id}_1$}] {N}
                                        ++(0:\widthletter) node[letter, label=below:{$\trm{id}_2$}] {J}
                                        ++(0:\widthletter) node[letter, label=below:{$\trm{id}_3$}] {O};
                                }
                                \newcommand\totoa[2]{
                                    \path
                                        #1
                                        ++#2
                                        node[letter, label=below:{}] {}
                                        ++(0:\widthletter) node[letter, label=below:{}] {}
                                        ++(0:\widthletter) node[letter, label=below:{}] {}
                                        ++(0:\widthletter) node[letter, label=below:{}] {}
                                        ++(0:\widthletter) node[letter, label=below:{}] {};
                                }
                                \newcommand\totob[2]{
                                    \path
                                        #1
                                        ++#2
                                        node[letter, label=below:{$\trm{id}_0$}] {B}
                                        ++(0:\widthletter) node[letter, label=below:{$?$}] {}
                                        ++(0:\widthletter) node[letter, label=below:{$\trm{id}_1$}] {N}
                                        ++(0:\widthletter) node[letter, label=below:{$\trm{id}_2$}] {J}
                                        ++(0:\widthletter) node[letter, label=below:{$\trm{id}_3$}] {O};
                                }
                                \newcommand\totoc[2]{
                                    \path
                                        #1
                                        ++#2
                                        node[letter, label=below:{$\trm{id}_0$}] {B}
                                        ++(0:\widthletter) node[letter, label=below:{$\trm{id}_{0.5}$}] {A}
                                        ++(0:\widthletter) node[letter, label=below:{$\trm{id}_1$}] {N}
                                        ++(0:\widthletter) node[letter, label=below:{$\trm{id}_2$}] {J}
                                        ++(0:\widthletter) node[letter, label=below:{$\trm{id}_3$}] {O};
                                }

                                \newcommand\offseta{ (90:1) }

                                \path
                                    node {\textbf{A}}
                                    ++(0:0.5) node (a) {}
                                    +(0:10) node (a-end) {}
                                    +(0:1) node[point] (a-initial) {}
                                    +(0:6) node (a-ins-a) {}
                                    +(0:9) node (a-final) {};

                                \initialstate{(a-initial)}{\offseta};
                                \draw[dotted] (a) -- (a-initial) (a-final) -- (a-end);
                                \only<7,8>{
                                    \draw[->, thick] (a-initial) -- (a-final);
                                }
                                \onslide<9->{
                                    \path (a-ins-a) node[point, label=-170:{$\trm{ins}(B \prec A \prec N)$}] {};
                                    \draw[->, thick] (a-initial) -- (a-ins-a) -- (a-final);
                                }
                                \only<9>{\totoa{(a-ins-a)}{\offseta}}
                                \only<10>{\totob{(a-ins-a)}{\offseta}}
                                \onslide<11->{\totoc{(a-ins-a)}{\offseta}}
                            \end{tikzpicture}
                        }
                    \end{figure}
                \end{block}
            }
        \end{column}
    \end{columns}
    \begin{itemize}
        \item<6-> Changements d'indices sont \alert{source de conflits}
        \item<7-> CRDTs assignent des \alert{identifiants de position} \cite{2009-treedoc-preguica} à chaque élément
        \item<7->
            Identifiants permettent d'\alert{ordonner les élements}
            \onslide<8->{
                \begin{equation*}
                    \trm{id}_0 \lid \trm{id}_1 \lid \trm{id}_2 \lid \trm{id}_3
                \end{equation*}
            }
        \vspace{-1.5em}
        \item<10->
            Identifiants appartiennent à un \alert{espace dense}
            \onslide<11->{
                \begin{equation*}
                    \trm{id}_0 \lid \trm{id}_{0.5} \lid \trm{id}_1
                \end{equation*}
            }
    \end{itemize}
    \vspace{-1em}
    \onslide<12->{
        \begin{center}
            \alert{Utilise LogootSplit \cite{2013-logootsplit} comme base}
        \end{center}
    }
\end{frame}

\begin{frame}{Identifiant LogootSplit}
    \metroset{block=transparent}
    \begin{block}{Identifiant}
        \begin{itemize}
            \item Composé d'\alert{un ou plusieurs tuples} de la forme
        \end{itemize}
        \vspace{3em}
        \begin{equation*}
            \tikzmarknode{pos}{pos}^{
                \tikzmarknode{nodeId}{nodeId}~\tikzmarknode{nodeSeq}{nodeSeq}
            }_{\tikzmarknode{offset}{offset}}
        \end{equation*}
        \begin{tikzpicture}[overlay,remember picture,>=stealth,nodes={align=left,inner ysep=1pt},<-]
            % For "pos"
            \path<2-> (pos.north) ++ (0,2em) node[anchor=south east,color=ucl1mdred] (legend-pos){\textbf{position (a-z)}};
            \draw<2-> [color=ucl1mdred](pos.north) |- ([xshift=-0.3ex,color=ucl1mdred] legend-pos.south west);
            % For "nodeId"
            \path<3-> (nodeId.north) ++ (0,2em) node[anchor=south west,color=ucl1mdblue] (legend-nodeid){\textbf{identifiant (A-Z) de l'auteur}};
            \draw<3-> [color=ucl1mdblue](nodeId.north) |- ([xshift=0.3ex,color=ucl1mdblue] legend-nodeid.south east);
            % For "nodeSeq"
            \path<4-> (nodeSeq.south) ++ (0,-2em) node[anchor=north west,color=ucl2dkpurple] (legend-nodeseq){\textbf{numéro de séquence (1-9)}};
            \draw<4-> [color=ucl2dkpurple](nodeSeq.south) |- ([xshift=0.3ex,color=ucl2dkpurple] legend-nodeseq.south east);
        \end{tikzpicture}
    \end{block}
    \onslide<5->{
        \begin{block}{Relation d'ordre $\lid$}
            \begin{itemize}
                \item Se base sur l'\alert{ordre lexicographique sur les éléments des tuples}
            \end{itemize}
        \end{block}
    }
    \onslide<6->{
        \begin{block}{Exemples}
            \begin{equation*}
                \id{
                    \only<8>{\mathbf{d}}
                    \only<6,7,9,10,11,12>{d}
                }{F5}{0}
                \onslide<7->{
                    \lid \id{
                        \only<7,9,10,11,12>{m}
                        \only<8>{\mathbf{m}}
                    }{C1}{
                        \only<7,8,9,11,12>{0}
                        \only<10>{\textbf{0}}
                    }
                }
                \onslide<9->{
                    \lid \id{m}{C1}{
                        \only<9,11,12>{1}
                        \only<10>{\textbf{1}}
                        }
                }
            \end{equation*}
            \onslide<11->{
                \begin{equation*}
                    \id{i}{B1}{0} \lid \only<11>{\quad?\quad} \only<12>{\id{i}{B1}{0}\alert{\id{f}{A1}{0}}} \lid \id{i}{B1}{1}
                \end{equation*}
            }
        \end{block}
    }
\end{frame}

\begin{frame}[fragile]{Bloc LogootSplit}
    \begin{itemize}
        \item Coûteux de stocker les identifiants de chaque élément
    \end{itemize}
    \begin{figure}[!ht]
        \begin{tikzpicture}
            \path
                node[letter, label=below:{$\id{m}{C1}{0}$}] {B}
                ++(0:\widthletter) node[letter, label=below:{$\id{m}{C1}{1}$}] {A}
                ++(0:\widthletter) node[letter, label=below:{$\id{m}{C1}{2}$}] {N}
                ++(0:\widthletter) node[letter, label=below:{$\id{m}{C1}{3}$}] {J}
                ++(0:\widthletter) node[letter, label=below:{$\id{m}{C1}{4}$}] {O};
        \end{tikzpicture}
    \end{figure}
    \pause
    \vspace{-1em}
    \begin{itemize}
        \item Aggrège en un \alert{bloc} les éléments ayant des \alert{identifiants contigus}
    \end{itemize}
    \begin{block}{Identifiants contigus}
        Deux identifiants sont contigus si et seulement si :
        \begin{enumerate}
            \item les deux identifiants sont identiques à l'exception de leur dernier offset
            \item ces deux derniers offsets sont consécutifs
        \end{enumerate}
    \end{block}
    \pause
    \begin{itemize}
        \item Note l'intervalle d'identifiants d'un bloc : $\id{pos}{nodeId~nodeSeq}{begin..end}$
    \end{itemize}
    \begin{figure}[!ht]
        \begin{tikzpicture}
            \path
                node[block, label=below:{$\id{m}{C1}{0..4}$}] {BANJO};
        \end{tikzpicture}
    \end{figure}
\end{frame}

\begin{frame}{Limites de LogootSplit}
    \metroset{block=transparent}
    \begin{block}{Taille du contenu comparée à la taille de la séquence LogootSplit}
        \begin{columns}
            \begin{column}{0.6\textwidth}
                \begin{figure}
                    \resizebox{\columnwidth}{!}{
                        \includegraphics{img/ls-vs-content-snapshot-sizes-7k5.pdf}
                    }
                \end{figure}
            \end{column}
            \begin{column}{0.4\textwidth}
                \pause
                \begin{block}{Constat}
                    \begin{itemize}
                        \item 1\% contenu\dots
                        \item \dots 99\% métadonnées
                    \end{itemize}
                \end{block}
                \pause
                \begin{center}
                    \alert{Et ça augmente !}
                \end{center}
                \pause
                \begin{block}{Impact}
                    \begin{itemize}
                        \item Surcoût \alert{mémoire}\dots
                        \item \dots mais aussi surcoût en \alert{calculs} et en \alert{bande-passante}
                    \end{itemize}
                \end{block}
            \end{column}
        \end{columns}
    \end{block}
\end{frame}

\begin{frame}[standout]
    Comment \alert{réduire le surcoût des CRDTs pour le type Séquence}, dans le cadre de \alert{systèmes pair-à-pair} ?
\end{frame}

% \begin{frame}[fragile]{Proposition}
%     \metroset{block=transparent}
%     \begin{block}{}
%         \begin{figure}
%             \resizebox{\textwidth}{!}{
%               \begin{tikzpicture}
%                 \newcommand\nodeo[1]{
%                     node[letter, label=#1:{$\trm{id}_1$}] {O}
%                 }
%                 \newcommand\noden[1]{
%                     node[letter, label=#1:{$\trm{id}_2$}] {N}
%                 }
%                 \newcommand\nodee[1]{
%                     node[letter, label=#1:{$\trm{id}_3$}] {E}
%                 }
%                 \newcommand\nodeebis[1]{
%                     node[letter, label=#1:{$\trm{id}_{97}$}] {E}
%                 }
%                 \newcommand\nodenbis[1]{
%                     node[letter, label=#1:{$\trm{id}_{98}$}] {N}
%                 }
%                 \newcommand\noded[1]{
%                     node[letter, label=#1:{$\trm{id}_{99}$}] {D}
%                 }


%                 \newcommand\renoneend[1]{
%                     node[block, label=#1:{$\trm{id}'_{begin..end}$}] {ONE$\cdots$END}
%                 }

%                 \newcommand\initialstate[3]{
%                     \path
%                     #1
%                     ++#2
%                     ++(0:0.5)
%                     ++(#3:0.5) \nodeo{-#3}
%                     ++(0:\widthletter) \noden{-#3}
%                     ++(0:\widthletter) \nodee{-#3}
%                     +(0:1.5 * \widthletter) node {$\cdots$}
%                     ++(0:2 * \widthletter) \nodeebis{-#3}
%                     ++(0:\widthletter) \nodenbis{-#3}
%                     ++(0:\widthletter) \noded{-#3};
%                 }

%                 \newcommand\finalstate[3]{
%                   \path
%                   #1
%                   ++#2
%                   ++(0:0.5)
%                   ++(#3:0.5) \renoneend{-#3};
%                 }

%                 \newcommand\offseta{ (90:0.7) }

%                 \path
%                     node {\textbf{A}}
%                     ++(0:0.5) node (a) {}
%                     +(0:11) node (a-end) {}
%                     +(0:1) node[point] (a-initial) {}
%                     +(0:8) node (a-ren) {}
%                     +(0:10) node (a-final) {};

%                 \initialstate{(a-initial)}{\offseta}{90};

%                 \draw[dotted] (a) -- (a-initial) (a-final) -- (a-end);

%                 \only<1,2>{
%                     \draw[->, thick] (a-initial) -- (a-final);
%                 }
%                 \onslide<2->{
%                     \finalstate{(a-ren)}{\offseta}{90};
%                 }
%                 \onslide<3->{
%                     \path (a-ren) node[point, label=-170:{$?$}] {};
%                     \draw[->, thick] (a-initial) --  (a-ren) -- (a-final);
%                 }
%               \end{tikzpicture}
%             }
%           \end{figure}
%           \begin{itemize}
%             \item \alert{Convertir l'état} actuel\dots
%             \item<2-> \dots \alert{en un état optimisé} (identifiants de taille minimale, moins de blocs)\dots
%             \item<3-> \dots à l'aide d'une \alert{nouvelle opération}
%           \end{itemize}
%     \end{block}
% \end{frame}

\subsection{RenamableLogootSplit}

\begin{frame}{Contribution : RenamableLogootSplit}
    \begin{itemize}
        \item CRDT pour le type Séquence qui incorpore un mécanisme de renommage
        \item Réassigne de nouveaux identifiants aux éléments via une nouvelle opération : \ren
    \end{itemize}
    \begin{block}{Propriétés de l'opération \ren}
        \begin{itemize}
            \item Est déterministe
            \item Préserve l'intention des utilisateur-rices
            \item Préserve les propriétés de la séquence, \ie l'unicité et l'ordre de ses identifiants
            \item Est commutative avec les opérations \ins, \rmv mais aussi \ren concurrentes
        \end{itemize}
    \end{block}
\end{frame}

\begin{frame}[fragile]{Opération \ren}
    \begin{figure}
      \resizebox{\textwidth}{!}{
        \begin{tikzpicture}
          \newcommand\nodeh[1]{
              node[letter, label=#1:{$\id{i}{B1}{0}$}] {H}
          }
          \newcommand\nodee[1]{
              node[letter, fill=\colorblockone, label=#1:{$\coloridone\id{i}{B1}{0}\id{f}{A1}{0}$}] {E}
          }
          \newcommand\nodelo[1]{
              node[block, label=#1:{$\id{i}{B1}{1..2}$}] {LO}
          }
          \newcommand\renh[1]{
              node[letter, fill=\colorblocktwo, label=#1:{$\coloridtwo\id{i}{A2}{0}$}] {H}
          }
          \newcommand\rene[1]{
              node[letter, fill=\colorblocktwo, label=#1:{$\coloridtwo\id{i}{A2}{1}$}] {E}
          }
          \newcommand\renl[1]{
              node[letter, fill=\colorblocktwo, label=#1:{$\coloridtwo\id{i}{A2}{2}$}] {L}
          }
          \newcommand\reno[1]{
              node[letter, fill=\colorblocktwo, label=#1:{$\coloridtwo\id{i}{A2}{3}$}] {O}
          }
          \newcommand\renhelo[1]{
              node[block, fill=\colorblocktwo, label=#1:{$\coloridtwo\id{i}{A2}{0..3}$}] {HELO}
          }

          \newcommand\initialstate[3]{
              \path
              #1
              ++#2
              ++(0:0.5)
              ++(#3:0.5) \nodeh{-#3}
              ++(0:\widthletter) \nodee{#3}
              ++(0:\widthletter) \nodelo{-#3};
          }

          \newcommand\stateh[3]{
              \path
              #1
              ++#2
              ++(0:0.5)
              ++(#3:0.5) \renh{-#3};
          }

          \newcommand\statehe[3]{
            \path
            #1
            ++#2
            ++(0:0.5)
            ++(#3:0.5) \renh{-#3}
            ++(0:\widthletter) \rene{-#3};
          }

          \newcommand\statehel[3]{
            \path
            #1
            ++#2
            ++(0:0.5)
            ++(#3:0.5) \renh{-#3}
            ++(0:\widthletter) \rene{-#3}
            ++(0:\widthletter) \renl{-#3};
          }

          \newcommand\statehelo[3]{
            \path
            #1
            ++#2
            ++(0:0.5)
            ++(#3:0.5) \renh{-#3}
            ++(0:\widthletter) \rene{-#3}
            ++(0:\widthletter) \renl{-#3}
            ++(0:\widthletter) \reno{-#3};
          }

          \newcommand\finalstate[3]{
            \path
            #1
            ++#2
            ++(0:0.5)
            ++(#3:0.5) \renhelo{-#3};
          }

          \newcommand\offseta{ (90:0.7) }

          \path
              node {\textbf{A}}
              ++(0:0.5) node (a) {}
              +(0:11) node (a-end) {}
              +(0:1) node[point] (a-initial) {}
              +(0:6) node[point] (a-ren-a1) {}
              +(0:10) node (a-final) {};

          \initialstate{(a-initial)}{\offseta}{90};

          \onslide<1-8>{\path (a-ren-a1) node[label=-170:{$\trm{ren}()$}, label={[xshift=0pt]-10:{$\trm{ren}(A,2)$}}] {};}

          \only<4>{\stateh{(a-ren-a1)}{\offseta}{90}};
          \only<5>{\statehe{(a-ren-a1)}{\offseta}{90};}
          \only<6>{\statehel{(a-ren-a1)}{\offseta}{90};}
          \only<7>{\statehelo{(a-ren-a1)}{\offseta}{90};}
          \onslide<8->{\finalstate{(a-ren-a1)}{\offseta}{90};}

          \draw[dotted] (a) -- (a-initial) (a-final) -- (a-end);
          \draw[->, thick] (a-initial) --  (a-ren-a1) -- (a-final);
        \end{tikzpicture}
      }
    \end{figure}
    \begin{itemize}
      \item<2-> Génère nouvel identifiant pour le 1er élément : \onslide<3->{$\id{i}{B1}{0} \to \id{i}{A2}{0}$}
      \item<4-> Puis génère identifiants contigus pour éléments suivants : \onslide<5->{$\id{i}{A2}{1}$}\onslide<6->{, $\id{i}{A2}{2}$}\onslide<7->{, \dots}
    \end{itemize}

    \onslide<8->{
      \begin{center}
        \alert{Regroupe tous les éléments en 1 unique bloc}
      \end{center}
    }
\end{frame}

\subsection{Intégration des opérations \ins et \rmv concurrentes}

\begin{frame}[fragile]{Interactions avec opérations \ins et \rmv concurrentes}
  \begin{figure}
    \resizebox{\textwidth}{!}{
      \begin{tikzpicture}
        \newcommand\nodeh[1]{
            node[letter, label=#1:{$\id{i}{B1}{0}$}] {H}
        }
        \newcommand\nodee[1]{
            node[letter, fill=\colorblockone, label=#1:{$\coloridone\id{i}{B1}{0}\id{f}{A1}{0}$}] {E}
        }
        \newcommand\nodelo[1]{
            node[block, label=#1:{$\id{i}{B1}{1..2}$}] {LO}
        }
        \newcommand\renhelo[1]{
            node[block, fill=\colorblocktwo, label=#1:{$\coloridtwo\id{i}{A2}{0..3}$}] {HELO}
        }
        \newcommand\nodel[1]{
            node[letter, fill=\colorblockthree, label=#1:{$\coloridthree\id{i}{B1}{0}\id{m}{B2}{0}$}] {L}
        }
        \newcommand\crossl[1]{
            node[letter, cross, fill=\colorblockthree, label=#1:{$\coloridthree\id{i}{B1}{0}\id{m}{B2}{0}$}] {L}
        }

        \newcommand\initialstate[3]{
            \path
                #1
                ++#2
                ++(0:0.5)
                ++(#3:0.5) \nodeh{-#3}
                ++(0:\widthletter) \nodee{#3}
                ++(0:\widthletter) \nodelo{-#3};
        }

        \newcommand\statehelo[3]{
            \path
                #1
                ++#2
                ++(0:0.5)
                ++(#3:0.5) \renhelo{-#3};
        }

        \newcommand\insl[3]{
            \path
            #1
            ++#2
            ++(0:0.5)
            ++(#3:0.5) \nodeh{-#3}
            ++(0:\widthletter) \nodee{#3}
            ++(0:\widthletter) \nodel{-#3}
            ++(0:\widthletter) \nodelo{#3};
        }

        \newcommand\finalstate[3]{
            \path
                #1
                ++#2
                ++(0:0.5)
                ++(#3:0.5) \renhelo{-#3}
                ++(0:\widthblock) \nodel{#3};
        }

        \newcommand\finalstatecrossed[3]{
            \path
                #1
                ++#2
                ++(0:0.5)
                ++(#3:0.5) \renhelo{-#3}
                ++(0:\widthblock) \crossl{#3};
        }

        \newcommand\offseta{ (90:0.7) }
        \newcommand\offsetb{ (-90:0.7) }

        \path
            node {\textbf{A}}
            ++(0:0.5) node (a) {}
            +(0:16) node (a-end) {}
            +(0:1) node[point] (a-initial) {}
            +(0:6) node[point, label=-170:{$\trm{ren}()$}, label={[xshift=0pt]-10:{$\trm{ren}(A,2)$}}] (a-ren-a1) {}
            +(0:12) node (a-recv-ins-l) {}
            +(0:15) node (a-final) {};
        \path<3->
            (a-recv-ins-l) node[point] {};

        \initialstate{(a-initial)}{\offseta}{90};
        \statehelo{(a-ren-a1)}{\offseta}{90};
        \only<4>{\finalstate{(a-recv-ins-l)}{\offseta}{90};}
        \onslide<5->{\finalstatecrossed{(a-recv-ins-l)}{\offseta}{90};}

        \draw[dotted] (a) -- (a-initial) (a-final) -- (a-end);
        \draw<-2>[->, thick] (a-initial) --  (a-ren-a1) -- (a-final);
        \draw<3->[->, thick] (a-initial) --  (a-ren-a1) -- (a-recv-ins-l) -- (a-final);

        \path
            ++(270:2) node {\textbf{B}}
            ++(0:0.5) node (b) {}
            +(0:16) node (b-end) {}
            +(0:1) node[point] (b-initial) {}
            +(0:6) node (b-ins-l) {}
            +(0:15) node (b-final) {};
        \path<2->
            (b-ins-l) node[point, label=170:{$\trm{ins}(E \prec L \prec L)$}] {};
        \path<3->
            (b-ins-l) node[label={[xshift=45pt]10:{$\trm{ins}({\coloridthree\id{i}{B1}{0}\id{m}{B2}{0}},L)$}}] {};

        \initialstate{(b-initial)}{\offsetb}{-90};
        \onslide<2->{\insl{(b-ins-l)}{\offsetb}{-90};}

        \draw[dotted] (b) -- (b-initial) (b-final) -- (b-end);
        \draw<1>[->, thick] (b-initial) -- (b-final);
        \draw<2->[->, thick] (b-initial) --  (b-ins-l) -- (b-final);

        \draw<3->[->, dashed, shorten >= 1] (b-ins-l) -- (a-recv-ins-l);
      \end{tikzpicture}
    }
  \end{figure}
  \begin{itemize}
    \item Noeuds peuvent générer opérations concurrentes aux opérations \ren
    \item<5-> Opérations produisent anomalies si intégrées naïvement
  \end{itemize}
  \onslide<6>{
    \begin{center}
      \alert{Nécessité d'un mécanisme dédié}
    \end{center}
  }
\end{frame}

\begin{frame}{Décomposition de la contribution}
    \begin{block}{Besoins}
        \begin{enumerate}
            \item Détecter les opérations concurrents aux opérations \ren
            \item Prendre en compte les effets des opérations \ren lors de l'intégration des opérations concurrentes
            \pause
            \item Résoudre les conflits provoqués par des opérations \ren concurrentes
            \pause
            \item Supprimer les métadonnées introduites par le mécanisme de renommage lui-même
        \end{enumerate}
    \end{block}
    \pause
    \begin{itemize}
        \item Implémentation disponible à l'adresse suivante : \url{https://github.com/coast-team/mute-structs}
    \end{itemize}
\end{frame}
